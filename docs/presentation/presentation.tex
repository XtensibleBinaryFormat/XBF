\documentclass{beamer}
\usepackage{amsfonts,amsmath,oldgerm}
\usepackage{bytefield}
\usepackage{listings}

% pympress notes page config
\iffalse
	\setbeamertemplate{note page}
	{
		\insertslideintonotes{0.01}
		\rule{\textwidth}{0.1pt}
		%\color{blue} \scriptsize
		\insertnote
	}
	\setbeameroption{show notes on second screen}
\fi

% adjust gigantic tabs in code blocks, not actually rust format
\lstdefinestyle{rust}{
	tabsize=2
}
\lstset{style=rust}

\usetheme{sintef}
\usefonttheme[onlymath]{serif}
\titlebackground*{assets/background}

% needed this a bunch so made a macro for myself
\newcommand{\rot}[1]{\rotatebox{90}{#1}}

\title{Xtensible Binary Format (XBF): An Efficient Self-Describing Binary Format}
\author{\href{mailto:dkrautha@pm.me}{David Krauthamer}}

\begin{document}
\maketitle

\section{Introduction}

\begin{frame}{Web Data Interchange Formats}
	\begin{itemize}
		\item Most common: JSON and XML
		\item Human-Readable and Self-Describing
		\item Alternatives: CSV, MessagePack, CBOR
	\end{itemize}
\end{frame}

\begin{frame}{Shortcomings of Current Formats}
	\begin{itemize}
		\item Inefficiency of plain text.
		\item Repeated sending of metadata.
		\item Use of big-endian.
	\end{itemize}
\end{frame}

\begin{frame}[fragile]{Plain Text Example}
	\begin{table}
		\caption{Sending A 32-bit Unsigned Integer}
		\begin{center}
			\begin{tabular}{|c|c|c|}
				\hline
				\textbf{Number} & \textbf{ASCII Bytes} & \textbf{Binary Bytes} \\
				\hline
				1               & 1                    & 4                     \\
				9999            & 4                    & 4                     \\
				7654321         & 7                    & 4                     \\
				\hline
			\end{tabular}
		\end{center}
	\end{table}
\end{frame}

\begin{frame}[fragile]{Repeated Metadata}
	\begin{columns}
		\begin{column}{0.5\textwidth}
			\begin{lstlisting}[language=XML]
<person>
	<name>John Jackson</name>
	<age>25</age>
</person>
\end{lstlisting}
		\end{column}
		\begin{column}{0.5\textwidth}
			\begin{table}
				\caption{XML Space Utilization}
				\center
				\begin{tabular}{|c|c|c|}
					\hline
					\textbf{} & \textbf{Bytes Used} & \textbf{Percent} \\
					\hline
					Metadata  & 45                  & 76\%             \\
					Data      & 14                  & 24\%             \\
					\hline
					Total     & 59                  & 100              \\
					\hline
				\end{tabular}
			\end{table}
		\end{column}
	\end{columns}
\end{frame}

\begin{frame}[fragile]{Big Endian Example: 32-bit Unsigned Integer}
	\begin{center}
		\textbf{Number To Send: 1027} \\
	\end{center}
	\begin{columns}
		\begin{column}{0.3\textwidth}
			\begin{bytefield}{16}
				\wordbox{1}{Sender (Little Endian)} \\
				\bitboxes{4}{{3} {4} {0} {0}} \\
			\end{bytefield}
		\end{column}
		\begin{column}{0.3\textwidth}
			\begin{bytefield}{16}
				\wordbox{1}{On the Wire (Big Endian)} \\
				\bitboxes{4}{{0} {0} {4} {3}} \\
			\end{bytefield}
		\end{column}
		\begin{column}{0.3\textwidth}
			\begin{bytefield}{16}
				\wordbox{1}{Recipient (Little Endian)} \\
				\bitboxes{4}{{3} {4} {0} {0}} \\
			\end{bytefield}
		\end{column}
	\end{columns}
\end{frame}

\begin{frame}{Improvements For A New Format}
	\begin{itemize}
		\item When possible, send metadata only a single time (mainly for homogenous lists).
		\item Allow sending metadata to be optional, such as if a client already has it.
		\item Utilize little-endian to eliminate endian conversions on common architectures.
		\item Prioritize a simple type system.
	\end{itemize}
\end{frame}

\section{Format Design}

\begin{frame}{Primitives and Primitive Metadata}
	\begin{columns}
		\begin{column}{0.5\textwidth}
			\begin{itemize}
				\item Boolean
				\item U8, U16, U32, U64, U128, U256
				      \begin{itemize}
					      \item Equivalent to uint*\_t types in C
				      \end{itemize}
				\item I8, I16, I32, I64, I128, I256
				      \begin{itemize}
					      \item Equivalent to int*\_t types in C
				      \end{itemize}
				\item F32, F64
				      \begin{itemize}
					      \item IEEE 754 floating point numbers
					      \item Equivalent to float and double in C
				      \end{itemize}
				\item Bytes  (Non-UTF-8 Bytes)
				\item String (UTF-8 Bytes)
			\end{itemize}
		\end{column}
		\begin{column}{0.25\textwidth}
			\center
			\begin{tabular}{|c|c|}
				\hline
				\textbf{Type} & \textbf{Value} \\
				\hline
				Boolean       & 0              \\
				U8            & 1              \\
				U16           & 2              \\
				U32           & 3              \\
				U64           & 4              \\
				U128          & 5              \\
				U256          & 6              \\
				\hline
			\end{tabular}
		\end{column}
		\begin{column}{0.25\textwidth}
			\center
			\begin{tabular}{|c|c|}
				\hline
				\textbf{Type} & \textbf{Value} \\
				\hline
				I8            & 7              \\
				I16           & 8              \\
				I32           & 9              \\
				I64           & 10             \\
				I128          & 11             \\
				I256          & 12             \\
				F32           & 13             \\
				F64           & 14             \\
				Bytes         & 15             \\
				String        & 16             \\
				\hline
			\end{tabular}
		\end{column}
	\end{columns}
\end{frame}

\begin{frame}[fragile]{Primitive Example}
	\begin{itemize}
		\item 16-Bit Unsigned Integer: 1024 \\
		      \begin{bytefield}{24}
			      \bitbox{8}{Disc} & \bitbox{16}{1024} \\
			      \bitbox{8}{2} & \bitboxes{8}{04} \\
		      \end{bytefield}
		\item String: "hello" \\
		      \begin{bytefield}[bitheight=\widthof{"Disc"}]{28}
			      \bitbox{2}{\rot{Disc}} & \bitbox{16}{length} & \bitboxes{2}{hello} \\
			      \bitboxes{2}{{16} {2} {0} {0} {0} {0} {0} {0} {0} {\rot{104}} {\rot{101}} {\rot{108}} {\rot{108}} {\rot{111}}}
		      \end{bytefield}

	\end{itemize}
\end{frame}

\begin{frame}{Vectors (Homogenous Lists)}
	Values
	\begin{itemize}
		\item Homogenous list of values that has a known length.
		\item Length is sent as a part of the type, not in metadata.
		\item Inner type Information is \emph{not} sent as a part of the value.
	\end{itemize}

	Metadata
	\begin{itemize}
		\item Single byte discriminant value is sent (String + 1).
		\item Followed by internal type metadata (can continue recursively).
		\item Length is \emph{not} sent as a part of the metadata.
	\end{itemize}
\end{frame}

\begin{frame}[fragile]{Vector Example}
	\begin{center}
		Vector of 16-Bit Unsigned Integers: [42, 1024] \vfill \break

		DV = Discriminant of a Vector

		D16 = Discriminant of a 16-Bit Unsigned Integer \vfill \break

		\begin{bytefield}{26}
			\bitbox{2}{\rot{DV}} & \bitbox{2}{\rot{D16}} & \bitbox{16}{length} & \bitboxes{4}{{42} {1024}} \\
			\bitboxes{2}{{17} {2} {2} {0} {0} {0} {0} {0} {0} {0} {42} {0} {0} {4}}
		\end{bytefield}
	\end{center}
\end{frame}

\begin{frame}{Structs (Aggregate Types)}
	\begin{itemize}
		\item	An aggregate type containing a name as well as named fields.
		\item May not contain duplicate field names.
		\item Fields are sent in sequence in the order they are listed in metadata.
		\item Serialized value does \emph{not} include:
		      \begin{itemize}
			      \item Struct name.
			      \item Number of fields.
			      \item Field names.
			      \item Field types.
		      \end{itemize}
	\end{itemize}
\end{frame}

\begin{frame}{Struct Metadata}
	\begin{itemize}
		\item Single byte discriminant value is sent (Vector + 1).
		\item Followed by Struct name.
		\item Next, number of fields as an unsigned 16-bit integer.
		\item Last, pairs of field names and types.
		      \begin{itemize}
			      \item May continue recursively with nested types of Structs or Vectors.
		      \end{itemize}
		\item All names are sent with the same format as primitive Strings.
	\end{itemize}
\end{frame}

\begin{frame}[fragile]{Struct Metadata Example}
	\begin{columns}
		\begin{column}{0.65\textwidth}
			\center
			\begin{bytefield}{32}
				\bitbox{2}{\rot{DS}} & \bitbox{16}{struct name length} & \bitboxes{2}{Point} & \bitbox{4}{fields} \\
				\bitboxes{2}{{18} {5} {0} {0} {0} {0} {0} {0} {0} {80} {\rot{111}} {\rot{105}} {\rot{110}} {\rot{116}} {2} {0}} \\
			\end{bytefield}
			\begin{bytefield}{32}
				\bitbox{16}{first name length} & \bitboxes{2}{first} & \bitbox{2}{\rot{DF}} \\
				\bitboxes{2}{{5} {0} {0} {0} {0} {0} {0} {0} {\rot{102}} {\rot{105}} {\rot{114}} {\rot{115}} {\rot{116}} {7}} \\
			\end{bytefield}
			\begin{bytefield}{32}
				\bitbox{16}{last name length} & \bitboxes{2}{last} & \bitbox{2}{\rot{DL}} \\
				\bitboxes{2}{{4} {0} {0} {0} {0} {0} {0} {0} {\rot{108}} {\rot{97}} {\rot{115}} {\rot{116}} {7}}
			\end{bytefield}
		\end{column}
		\begin{column}{0.35\textwidth}
			\begin{lstlisting}
struct Point {
	first: i8,
	last: i8
}
\end{lstlisting}
			\hfill \break
			DS = Struct Discriminant

			DF = Field "first" discriminant

			DL = Field "last" discriminant
		\end{column}
	\end{columns}
\end{frame}

\begin{frame}[fragile]{Struct Value Example}
	\begin{columns}
		\begin{column}{0.5\textwidth}
			\center
			\begin{bytefield}{12}
				\bitbox{6}{first} & \bitbox{6}{last} \\
				\bitbox{6}{42} & \bitbox{6}{251} \\
			\end{bytefield}
		\end{column}
		\begin{column}{0.5\textwidth}
			\center
			\begin{lstlisting}
let my_point = Point {
	first: 42,
	last: -5,
};
			\end{lstlisting}
		\end{column}
	\end{columns}
\end{frame}

\section{Evaluation}

\begin{frame}{Encoder and Decoder Reference Implementation}
	\begin{itemize}
		\item Written in Rust
		      \begin{itemize}
			      \item Efficient code generation similar to C and C++
			      \item Built-in tagged unions
			      \item Strong memory safety guarantees
		      \end{itemize}
		\item Extensively tested
		      \begin{itemize}
			      \item Unit and integration tests
			      \item 100 percent code coverage
		      \end{itemize}
	\end{itemize}
\end{frame}

\begin{frame}{Performance Test}
	\begin{columns}
		\begin{column}{0.5\textwidth}
			Multi-threaded Server
			\begin{itemize}
				\item Downloads and parses 1 year of Sony stock history.
				\item Records the original CSV file size in bytes.
				\item Records the size in bytes when stored natively in memory.
				\item Waits to receive requests.
			\end{itemize}
		\end{column}
		\begin{column}{0.5\textwidth}
			Client
			\begin{itemize}
				\item Formats: CSV, MessagePack, CBOR, JSON, XML, XBF
				\item For each data format:
				      \begin{itemize}
					      \item Client sends 100 requests to the server.
					      \item Records average time to receive a response.
					      \item Records number of bytes read.
				      \end{itemize}
			\end{itemize}
		\end{column}
	\end{columns}
\end{frame}

\section{Results and Conclusions}

\begin{frame}[fragile]{Data}
	\begin{center}
		\begin{tabular}{|c|c|c|c|c|}
			\hline
			\textbf{Format} & \textbf{Avg Time (ms)} & \textbf{Bytes Read} & \textbf{Overhead (bytes)} & \textbf{Percent Overhead} \\
			\hline
			CSV             & 18.93                  & 16,411              & 1,823                     & 11.1                      \\
			MessagePack     & 11.22                  & 15,565              & 977                       & 6.22                      \\
			CBOR            & 16.95                  & 25,507              & 10,919                    & 42.8                      \\
			JSON            & 21.91                  & 31,180              & 16,592                    & 53.2                      \\
			XML             & 21.87                  & 43,699              & 29,111                    & 66.6                      \\
			XBF             & 11.32                  & 14,686              & 98                        & 0.67                      \\
			\hline
		\end{tabular}
	\end{center}
	Original CSV data size recorded by the server: 17,160 bytes.

	Native data size recorded by the server: 14,558 bytes.
\end{frame}

\begin{frame}{CSV Size Difference}
	CSV Size Difference
	\begin{itemize}
		\item Original data always used 8 digits of precision.
		\item CSV library used only writes the required digits of precision.
		\item Example: 83.500000 becomes 83.5
	\end{itemize}
\end{frame}

\begin{frame}{Time Performance}
	\begin{itemize}
		\item XBF was faster than every format except MessagePack.
		\item Formats that operated in binary were faster, even when sending more data (CBOR).
		\item XBF reference implementation was optimized for memory usage, more could be done for speed.
	\end{itemize}
\end{frame}

\begin{frame}{Size Performance}
	\begin{itemize}
		\item XBF required the least amount of bytes.
		\item Indicates metadata deduplication worked as intended.
		\item XBF can be even smaller with optimization techniques from MessagePack
		      \begin{itemize}
			      \item Coercion to smaller integer sizes when possible.
			      \item Differently sized variable length types.
		      \end{itemize}
		\item Could be possible to have a "negative" overhead.
	\end{itemize}
\end{frame}

\begin{frame}{Use Cases}
	\begin{itemize}
		\item Internet of Things and other low bandwidth scenarios.
		\item High volume applications.
	\end{itemize}
\end{frame}

\backmatter[notitle]
\end{document}
