\documentclass{beamer}
\usepackage{amsfonts,amsmath,oldgerm}
\usepackage{bytefield}
\usepackage{listings}

\lstdefinestyle{rust}{
	tabsize=2
}
\lstset{style=rust}

\usetheme{sintef}
\usefonttheme[onlymath]{serif}
\titlebackground*{assets/background}

\newcommand{\rot}[1]{\rotatebox{90}{#1}}

\title{Implementing XBF: An Efficient Self-Describing Binary Format}
% \subtitle{Using \LaTeX\ to prepare slides}
% \course{Master's Degree in Computer Science}
\author{\href{mailto:dkrautha@pm.me}{David Krauthamer}}
% \IDnumber{1234567}

\setbeameroption{show notes}

\begin{document}
\maketitle

\section{Introduction}

\begin{frame}{Web Data Interchange Formats}
	\begin{itemize}
		\item Human-Readable and Self-Describing
		      \note[item]{Most formats on the web are human-readable and self-describing}
		      \note[item]{Self describing means they include metadata to describe themselves, and don't require an external schema}
		      \note[item]{Human readable means plain text, ASCII or utf-8}
		\item Most common: JSON and XML
		      \note[item]{The most common of these formats is JSON and XML}
		\item Alternatives: CSV, MessagePack, CBOR
		      \note[item]{There are other human-readable formats such as CSV}
		      \note[item]{And others, such as CBOR (concise binary object representation), and MessagePack that are still self-describing, but use a binary format instead}
	\end{itemize}
\end{frame}

\begin{frame}{Shortcomings of Current Formats}
	\begin{itemize}
		\item Repeated sending of metadata.
		\item Inefficiency of plain text.
		      \note[item]{Expensive conversion between plain text and binary}
		      \note[item]{Requires more bytes to store the same data}
		\item Use of big-endian.
		      \note[item]{Most modern processors (x86 and ARM) are little-endian}
		      \note[item]{Both CBOR and MessagePack are big-endian}
	\end{itemize}
	% XML from paper here
\end{frame}

% possibly add three slides explaining each of these
% 1: XML being huge
% 2: plain text example from paper
% 3: 1027 as u32 0 4 3 0 -> 0 3 4 0 

\begin{frame}{Improvements For a New Format}
	\begin{itemize}
		\item When possible, send metadata only a single time (mainly for homogenous lists).
		\item Allow sending metadata to be optional, such as if a client already has it.
		      \note[item]{Keep metadata and data separate, not coupled together}
		\item Utilize little-endian to eliminate endian conversions on common architectures.
		\item Prioritize a simple type system.
		      \note[item]{Hopefully makes the implementation simpler}
	\end{itemize}
\end{frame}

\section{Format Design}

\begin{frame}{Primitives}
	\begin{itemize}
		\item Boolean
		      \note[item]{True of False, one byte}
		      \note[item]{All integers are little-endian}
		      \note[item]{Integer types are equivalent to the types in stdint.h C headers}
		\item U8, U16, U32, U64, U128, U256
		      \note[item]{Unsigned Integer Types}
		      \begin{itemize}
			      \item Equivalent to uint*\_t types in C
		      \end{itemize}
		\item I8, I16, I32, I64, I128, I256
		      \note[item]{Signed Integer Types}
		      \begin{itemize}
			      \item Equivalent to int*\_t types in C
		      \end{itemize}
		\item F32, F64
		      \note[item]{Floating Point Types}
		      \begin{itemize}
			      \item IEEE 754 floating point numbers
			      \item Equivalent to float and double in C
		      \end{itemize}
		\item Bytes  (Non-UTF-8 Bytes)
		      \note[item]{String but doesn't have to be UTF-8}
		\item String (UTF-8 Bytes)
		      \note[item]{Sends length as u64 followed by UTF-8 bytes}
	\end{itemize}
\end{frame}

\begin{frame}{Primitive Metadata}
	\center
	\begin{tabular}{|c|c|}
		\hline
		\textbf{Type} & \textbf{Discriminant} \\
		\hline
		Boolean       & 0                     \\
		U8            & 1                     \\
		U16           & 2                     \\
		U32           & 3                     \\
		...           & ...                   \\
		I256          & 12                    \\
		F32           & 13                    \\
		F64           & 14                    \\
		Bytes         & 15                    \\
		String        & 16                    \\
		\hline
	\end{tabular}
	% two columsn for each half of the list
	\note[item]{Metadata is sent as a single byte discriminant value}
	\note[item]{The discriminant begin at 0 with Boolean, and increase by 1 for each type}
	\note[item]{Progress in the same order as the previous slide, with String being the final value}
\end{frame}

\begin{frame}[fragile]{Primitive Example}
	\begin{itemize}
		\item 16-Bit Unsigned Integer: 1024 \\
		      \begin{bytefield}{24}
			      \bitbox{8}{Disc} & \bitbox{16}{1024} \\
			      \bitbox{8}{2} & \bitboxes{8}{04} \\
		      \end{bytefield}
		\item String: "hello" \\
		      \begin{bytefield}[bitheight=\widthof{"Disc"}]{28}
			      \bitbox{2}{\rot{Disc}} & \bitbox{16}{length} & \bitboxes{2}{hello} \\
			      \bitboxes{2}{{16} {2} {0} {0} {0} {0} {0} {0} {0} {\rot{104}} {\rot{101}} {\rot{108}} {\rot{108}} {\rot{111}}}
		      \end{bytefield}

	\end{itemize}
\end{frame}

\begin{frame}{Vectors (Homogenous Lists)}
	\begin{itemize}
		\item Homogenous list of values that has a known, variable length.
		\item Length is sent as a part of the type, not in metadata.
		      \note[item]{Length is an u64, the same as String and Bytes}
		\item Length is sent, followed by the elements.
		      \note[item]{No type info about elements, that's in metadata}
		\item Inner type Information is \emph{not} sent as a part of the value.
	\end{itemize}
\end{frame}

\begin{frame}{Vector Metadata}
	\begin{itemize}
		\item Single byte discriminant value is sent (String + 1).
		\item Followed by internal type metadata (can continue recursively).
		\item Length is \emph{not} sent as a part of the metadata.
	\end{itemize}
\end{frame}

\begin{frame}[fragile]{Vector Example}
	\begin{center}
		\textbf{Vector of 16-Bit Unsigned Integers:} [42, 1024]

		DV = Discriminant of a Vector

		D16 = Discriminant of a 16-Bit Unsigned Integer

		\begin{bytefield}{26}
			\bitbox{2}{\rot{DV}} & \bitbox{2}{\rot{D16}} & \bitbox{16}{length} & \bitboxes{4}{{42} {1024}} \\
			\bitboxes{2}{{17} {2} {2} {0} {0} {0} {0} {0} {0} {0} {42} {0} {0} {4}}
		\end{bytefield}
	\end{center}
\end{frame}

\begin{frame}{Structs (Aggregate Types)}
	\begin{itemize}
		\item	An aggregate type containing a name as well as named fields.
		\item May not contain duplicate field names.
		\item Fields are sent in sequence in the order they are listed in metadata.
		\item Serialized value does \emph{not} include:
		      \begin{itemize}
			      \item Struct name.
			      \item Number of fields.
			      \item Field names.
			      \item Field types.
		      \end{itemize}
	\end{itemize}
\end{frame}

\begin{frame}{Struct Metadata}
	\begin{itemize}
		\item Single byte discriminant value is sent (Vector + 1).
		\item Followed by Struct name.
		\item Next, number of fields as an unsigned 16-bit integer.
		      \note[item]{A 16-bit integer is used because it's very unlikely for a struct of more than 65 thousand fields to be necessary.}
		\item Last, pairs of field names and types.
		      \begin{itemize}
			      \item May continue recursively with nested types of Structs or Vectors.
		      \end{itemize}
		\item All names are sent with the same format as primitive Strings.
	\end{itemize}
\end{frame}

\begin{frame}[fragile]{Struct Metadata Example}
	\begin{columns}
		\begin{column}{0.65\textwidth}
			\note[item]{ DS = Discriminant of a Struct

				DX = Discriminant of field "first"

				DY = Discriminant of field "last"}
			\center
			\begin{bytefield}{32}
				\bitbox{2}{\rot{DS}} & \bitbox{16}{struct name length} & \bitboxes{2}{Point} & \bitbox{4}{fields} \\
				\bitboxes{2}{{18} {5} {0} {0} {0} {0} {0} {0} {0} {80} {\rot{111}} {\rot{105}} {\rot{110}} {\rot{116}} {2} {0}} \\
			\end{bytefield}
			\begin{bytefield}{32}
				\bitbox{16}{first name length} & \bitboxes{2}{first} & \bitbox{2}{\rot{DX}} \\
				\bitboxes{2}{{5} {0} {0} {0} {0} {0} {0} {0} {\rot{102}} {\rot{105}} {\rot{114}} {\rot{115}} {\rot{116}} {7}} \\
			\end{bytefield}
			\begin{bytefield}{32}
				\bitbox{16}{last name length} & \bitboxes{2}{last} & \bitbox{2}{\rot{DY}} \\
				\bitboxes{2}{{4} {0} {0} {0} {0} {0} {0} {0} {\rot{108}} {\rot{97}} {\rot{115}} {\rot{116}} {7}}
			\end{bytefield}
		\end{column}
		\begin{column}{0.35\textwidth}
			\begin{lstlisting}
struct Point {
	first: i8,
	last: i8
}
\end{lstlisting}
		\end{column}
	\end{columns}
	% add explanation for DS, DX, DY
\end{frame}

\begin{frame}[fragile]{Struct Value Example}
	\begin{columns}
		\begin{column}{0.5\textwidth}
			\center
			\begin{bytefield}{12}
				\bitbox{6}{first} & \bitbox{6}{last} \\
				\bitbox{6}{42} & \bitbox{6}{251} \\
			\end{bytefield}
		\end{column}
		\begin{column}{0.5\textwidth}
			\center
			\begin{lstlisting}
let my_point = Point {
	first: 42,
	last: -5,
};
			\end{lstlisting}
		\end{column}
	\end{columns}
\end{frame}

\section{Evaluation}

\begin{frame}{Tested Formats}
	\begin{itemize}
		\item CSV
		\item MessagePack
		\item CBOR
		\item JSON
		\item XML
	\end{itemize}
\end{frame}

\section{Results}

\begin{frame}{Average Time and Bytes Read}
	\center
	\begin{tabular}{|c|c|c|}
		\hline
		\textbf{Format} & \textbf{Avg Time (ms)} & \textbf{Bytes Read} \\
		\hline
		CSV             & 18.93                  & 16,411              \\
		MessagePack     & 11.22                  & 15,565              \\
		CBOR            & 16.95                  & 25,507              \\
		JSON            & 21.91                  & 31,180              \\
		XML             & 21.87                  & 43,699              \\
		XBF             & 11.32                  & 14,686              \\
		\hline
	\end{tabular}
\end{frame}

\begin{frame}{Overhead}
	\center
	\begin{tabular}{|c|c|c|}
		\hline
		\textbf{Format} & \textbf{Overhead (bytes)} & \textbf{Percent Overhead} \\
		\hline
		CSV             & 1,823                     & 11.1                      \\
		MessagePack     & 977                       & 6.22                      \\
		CBOR            & 10,919                    & 42.8                      \\
		JSON            & 16,592                    & 53.2                      \\
		XML             & 29,111                    & 66.6                      \\
		XBF             & 98                        & 0.67                      \\
		\hline
	\end{tabular}
\end{frame}

\backmatter[notitle]
\end{document}
